\section{{Resultados}}

Como se mencionó anteriormente, se realizaron dos instancias de experimentación.
En una primera instancia se consideraron únicamente antisacadas (N =
{first__starting_sample__subjects_count} sujetos) y en una segunda
instancia se incluyeron tanto antisacadas como prosacadas (N =
{second__starting_sample__subjects_count} sujetos).

\subsection{{Primera instancia}}

En la primera instancia se realizaron únicamente antisacadas. Se obtuvieron un
total de {first__starting_sample__trials_count} ensayos provenientes de
{first__starting_sample__subjects_count} sujetos.
Luego de aplicar el pre-procesamiento, la cantidad de ensayos se redujo a
{first__inlier_sample__trials_count} provenientes de
{first__inlier_sample__subjects_count} sujetos.
De estos, {first__correct_sample__trials_count} obtuvieron respuesta correcta,
{first__incorrect_sample__trials_count} obtuvieron respuesta incorrecta y
{first__without_response_sample__trials_count} no obtuvieron respuesta.
De las respuestas incorrectas, {first__corrected_sample__trials_count} tuvieron
una segunda sacada correctiva antes del fin del ensayo, es decir que tras mirar
al punto incorrecto, corrigió y realizó una sacada hacia el punto correcto.
Los ensayos correctos tuvieron un tiempo de respuesta mayor al de las
incorrectas (promedio $\pm$ desvio std: Correctas $=$ 
{first__correct_sample__mean_response_time} ms $\pm$ 
{first__correct_sample__stdev_response_time}; Incorrectas $=$
{first__incorrect_sample__mean_response_time} ms $\pm$
{first__incorrect_sample__stdev_response_time}).

{first__ages_distribution_figure}

{first__response_times_distribution_figure}

{first__disaggregated_antisaccades_figure}

\subsubsection{{Estimaciones desviadas}} \label{{section:results:skewed_estimates}}

Se encontró que para varios sujetos las estimaciones obtenidas por el prototipo
sobre el eje horizontal no coincidían con las posiciones reales de la mirada.
En cambio para cada uno de estos sujetos se detectó una desviación de sus
estimaciones durante todo el experimento.
La Figura \ref{{fig:skewed-estimations-example}} ilustra este fenómeno:
durante la fase de fijación para los sujetos 47 y 24 se obtienen estimaciones
respectivamente cercanas a los valores 2100 y 1400 píxeles, cuando los valores
reales serían 1100 y 900.
Esto no ocurre para todos los sujetos, como puede verse con los sujetos 43 y 22.

Para cada sujeto las estimaciones obtenidas son sin embargo consistentes en
cuánto a su desviación. Si bien los valores obtenidos no coinciden con las
coordenadas reales de los estímulos, sí se mantendrá el posicionamiento
relativo de las estimaciones durante la duración del experimento. Entonces, es
posible continuar identificando las regiones de interés (izquierda, centro y
derecha) del experimento. Estas desviaciones tuvieron que tenerse en cuenta al
momento de normalizar los datos. Asimismo, el hecho que se mantuviera el
correcto posicionamiento relativo dió lugar al mecanismo de validación
implementado.

% TODO: Figura estimaciones desviadas

\subsection{{Segunda instancia}}

% TODO: Write summary
