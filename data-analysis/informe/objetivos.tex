\chapter{Objetivos} 

En el presente trabajo nos proponemos desarrollar una herramienta que permita
realizar estudios clínicos remotos basados en \eyetracking.  Para guiar la
definición de nuestro caso de uso se realizarán experimentos en base a la tarea
de antisacadas.  Esto permitirá generar algunas métricas para evaluar la
calidad de la herramienta implementada y su capacidad en atacar problemas
habitualmente resueltos en contextos presenciales de laboratorio.

Inicialmente se evaluará la aplicabilidad de implementaciones existentes a
nuestro caso de uso.  Para ello tendrá que comprenderse el objetivo con el cual
fueron implementadas, su estado de mantenimiento y su capacidad en ser
modificadas o extendidas a nuestras necesidades.

Se integrarán luego aquellas herramientas que nos ayuden a cumplir nuestro
objetivo, o parte de ellas, extendiéndolas o combinándolas con desarrollos
propios de los módulos que faltaran.

Finalmente se estudiarán las limitaciones de la herramienta construida y,
habiendo realizado algunas pruebas experimentales, se propondrá un protocolo de
utilización que garantice su correcto funcionamiento.

En concreto se proponen los siguientes objetivos particulares:

\begin{itemize}
  \item Evaluar implementaciones existentes \eyetracking remoto y su
    aplicabilidad a nuestros casos de uso.
  \item Integrar dichas herramientas o las partes de ellas que efectivamente
    puedan ayudarnos a nuestro objetivo.
  \item Desarrollar los módulos que hagan falta apuntando a lograr una solución
    extensible.
  \item Recolectar datos en base a la tarea de antisacadas.
  \item Proponer un protocolo de utilización que garantice un funcionamiento
    correcto de la herramienta.
\end{itemize}
