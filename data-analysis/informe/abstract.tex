\chapter*{\tituloTesis}

\noindent
Partiendo de una implementación existente de modelado de la mirada se construyó
un prototipo de \eyetracker web orientado a realizar análisis clínicos remotos.
Se realizaron luego con él dos instancias de experimentación, utilizando como
caso de estudio la tarea de antisacadas.
En ella el sujeto debe contener el reflejo de mirar un estímulo visual que
aparece en algunos de los dos costados de la pantalla
Existe para ella diversa bibliografía sobre los procesos cognitivos
involucrados, sobre sus resultados esperados en poblaciones de distinta
condición neuropsicológica y sobre su potencial uso para asistir al diagnóstico
temprano de tales condiciones.
Se destaca además la novedad de herramientas web que cumplan esta función así
como el interés en su existencia por parte de la comunidad.
Conclusiones generales pudieron ser replicadas (\eg, las instancias correctas
resultan en promedio en un mayor tiempo de respuesta que aquellas incorrectas).
Sin embargo, el desempeño del prototipo queda muy por debajo que aquel de
\eyetrackers profesionales de laboratorio.
Esto pone en duda la posibilidad de resolver los mismos problemas.
Queda pendiente estudiar qué niveles de precisión son alcanzables con
\eyetracking en un contexto de navegador, así como estudiar en mayor
profundidad cuáles son los mecanismos apropiados para modelado de la mirada,
para calibrar, validar y detectar descalibraciones del sistema y para procesar
finalmente los datos resultantes.

\bigskip

\noindent \textbf{Palabras claves}:
eye tracking, navegador web, estudio clínico remoto, asistencia al diagnóstico,
antisacadas
